\chapter{Two Qubit Exchange Scattering}
Throughout this section, $\ket{\psi}$
will denote an arbitrary one qubit state.

\claim (Exchange scattering via Exchanger)

For any $z\in Bool$,

\grayeq{
\beq
\sqrt{2}
\begin{array}{c}
\Qcircuit @C=1em @R=.5em @!R{
\freegate{\bra{z}}
&\gate{H}
&\uarrowgate\qwx[1]
&\gate{\ket{\psi}}
\\
&\qw
&\darrowgate
&\gate{\ket{0}}
}
\end{array}
=
\begin{array}{c}
\Qcircuit @C=1em @R=.5em @!R{
&
\\
& \gate{\ket{\psi}}
}
\end{array}
\;.
\label{eq-ex-scat-via-exch}
\eeq}
\proof
Let LHS and RHS stand for the left and
right hand sides of
Eq.(\ref{eq-ex-scat-via-exch}).

\beq
LHS =
\sqrt{2}
\begin{array}{c}
\Qcircuit @C=1em @R=.5em @!R{
\freegate{\bra{z}}
&\gate{H}
&\gate{\ket{0}}
\\
&\qw
&\gate{\ket{\psi}}
}
\end{array}
= RHS
\;.
\eeq
\qed

\claim (Exchange scattering via CNOT)

For any $z\in Bool$,

\grayeq{
\beq
\sqrt{2}
\begin{array}{c}
\Qcircuit @C=1em @R=.5em @!R{
\freegate{\bra{z}}
&\qw
&\gate{H}
&\dotgate\qwx[1]
&\gate{\ket{\psi}}
\\
&\gate{\sigz^z}
&\qw
&\timesgate
&\gate{\ket{0}}
}
\end{array}
=
\begin{array}{c}
\Qcircuit @C=1em @R=.5em @!R{
&
\\
& \gate{\ket{\psi}}
}
\end{array}
\;.
\label{eq-ex-scat-via-cnot1}
\eeq}
\proof
Let LHS and RHS stand for the left and
right hand sides of
Eq.(\ref{eq-ex-scat-via-cnot1}).

\beqa
LHS &=&
\sqrt{2}
\begin{array}{c}
\Qcircuit @C=1em @R=.5em @!R{
\freegate{\bra{z}}
&\dotgate\qwx[1]
&\gate{H}
&\dotgate\qwx[1]
&\uarrowgate\qwx[1]
&\gate{\ket{0}}
\\
&\dotgate
&\qw
&\timesgate
&\darrowgate
&\gate{\ket{\psi}}
}
\end{array}\\
&=&
\sqrt{2}
\begin{array}{c}
\Qcircuit @C=1em @R=.5em @!R{
\freegate{\bra{z}}
&\gate{H}
&\timesgate\qwx[1]
&\dotgate\qwx[1]
&\dotgate\qwx[1]
&\timesgate\qwx[1]
&\dotgate\qwx[1]
&\gate{\ket{0}}
\\
&\qw
&\dotgate
&\timesgate
&\timesgate
&\dotgate
&\timesgate
&\gate{\ket{\psi}}
}
\end{array}\\
&=&
\sqrt{2}
\begin{array}{c}
\Qcircuit @C=1em @R=.5em @!R{
\freegate{\bra{z}}
&\gate{H}
&\gate{\ket{0}}
\\
&\qw
&\gate{\ket{\psi}}
}
\end{array}\\
&=& RHS
\;.
\eeqa
\altproof
For any $a\in Bool$:

\beq
\sqrt{2}
\begin{array}{c}
\Qcircuit @C=1em @R=.5em @!R{
\freegate{\bra{0}H}
&\dotgate\qwx[1]
&\gate{\ket{a}}
\\
&\timesgate
&\gate{\ket{0}}
}
\end{array}
=
\sigx^a(\bitb)\ket{0}_\bitb
= \ket{a}_\bitb
\;.
\eeq
Thus, for an arbitrary state $\ket{\psi}$,

\beq
\sqrt{2}
\begin{array}{c}
\Qcircuit @C=1em @R=.5em @!R{
\freegate{\bra{0}H}
&\dotgate\qwx[1]
&\gate{\ket{\psi}}
\\
&\timesgate
&\gate{\ket{0}}
}
\end{array}
=
\begin{array}{c}
\Qcircuit @C=1em @R=.5em @!R{
&
\\
&\gate{\ket{\psi}}
}
\end{array}
\;.
\label{eq-ex-scat-lite}
\eeq
In Eq.(\ref{eq-ex-scat-lite}),
if we multiply ket $\ket{\psi}$
by a pre-processing and a post-processing $\sigz^z$,
then we obtain

\beq
\sqrt{2}
\begin{array}{c}
\Qcircuit @C=1em @R=.5em @!R{
\freegate{\bra{0}H}
&\qw
&\dotgate\qwx[1]
&\gate{\sigz^z\ket{\psi}}
\\
&\gate{\sigz^z}
&\timesgate
&\gate{\ket{0}}
}
\end{array}
=
\begin{array}{c}
\Qcircuit @C=1em @R=.5em @!R{
&
\\
&\gate{(\sigz^z)^2\ket{\psi}}
}
\end{array}
\;,
\eeq
which easily yields

\beq
\sqrt{2}
\begin{array}{c}
\Qcircuit @C=1em @R=.5em @!R{
\freegate{\bra{z}}
&\gate{H}
&\dotgate\qwx[1]
&\gate{\ket{\psi}}
\\
&\gate{\sigz^z}
&\timesgate
&\gate{\ket{0}}
}
\end{array}
=
\begin{array}{c}
\Qcircuit @C=1em @R=.5em @!R{
&
\\
&\gate{\ket{\psi}}
}
\end{array}
\;.
\eeq
\qed


\claim (Another example of exchange scattering via CNOT)

For any $x\in Bool$,

\grayeq{
\beq
\sqrt{2}
\begin{array}{c}
\Qcircuit @C=1em @R=.5em @!R{
\freegate{\bra{x}}
&\qw
&\timesgate\qwx[1]
&\qw
&\gate{\ket{\psi}}
\\
&\gate{\sigx^x}
&\dotgate
&\gate{H}
&\gate{\ket{0}}
}
\end{array}
=
\begin{array}{c}
\Qcircuit @C=1em @R=.5em @!R{
&
\\
& \gate{\ket{\psi}}
}
\end{array}
\;.
\label{eq-ex-scat-via-cnot2}
\eeq}
\proof
In Eq.(\ref{eq-ex-scat-via-cnot1}),
if we replace $z$ by $x$ and
multiply the ket $\ket{\psi}$
by a pre-processing and a
post-processing $H$, then we obtain:

\beq
\sqrt{2}
\begin{array}{c}
\Qcircuit @C=1em @R=.5em @!R{
\freegate{\bra{x}}
&\qw
&\dotgate\qwx[1]
&\gate{H}
&\dotgate\qwx[1]
&\gate{H\ket{\psi}}
\\
&\gate{H}
&\dotgate
&\qw
&\timesgate
&\gate{\ket{0}}
}
\end{array}
=
\begin{array}{c}
\Qcircuit @C=1em @R=.5em @!R{
&
\\
& \gate{H^2\ket{\psi}}
}
\end{array}
\;.
\eeq
The last identity simplifies to

\beq
\sqrt{2}
\begin{array}{c}
\Qcircuit @C=1em @R=.5em @!R{
\freegate{\bra{x}}
&\dotgate\qwx[1]
&\timesgate\qwx[1]
&\qw
&\gate{\ket{\psi}}
\\
&\timesgate
&\dotgate
&\gate{H}
&\gate{\ket{0}}
}
\end{array}
=
\begin{array}{c}
\Qcircuit @C=1em @R=.5em @!R{
&
\\
& \gate{\ket{\psi}}
}
\end{array}
\;,
\eeq
which is the same as the claim that we
set out to prove.
\qed

\claim (Exchange scattering via a 2 qubit
projective measurement)

For any $j,k\in Bool$,

\grayeq{
\beq
2
\begin{array}{c}
\Qcircuit @C=1em @R=.5em @!R{
\freegate{\bra{j}}
&\gate{H}
&\multigate{1}{\pizz{k}}
&\qw
&\gate{\ket{\psi}}
\\
&\gate{\sigz^j\sigx^k}
&\ghost{\pizz{k}}
&\gate{H}
&\gate{\ket{0}}
}
\end{array}
=
\begin{array}{c}
\Qcircuit @C=1em @R=.5em @!R{
&
\\
& \gate{\ket{\psi}}
}
\end{array}
\;.
\label{eq-ex-scat-via-proj}
\eeq}
\proof
\beqa
\lefteqn{
\begin{array}{c}
\Qcircuit @C=1em @R=.5em @!R{
\freegate{\bra{j}}
&\gate{H}
&\multigate{1}{\pizz{k}}
&\qw
&\gate{\ket{\psi}}
\\
&\qw
&\ghost{\pizz{k}}
&\gate{H}
&\gate{\ket{0}}
}
\end{array}
=}
\label{eq-scat-one-box}
\\
&=&
\begin{array}{c}
\Qcircuit @C=1em @R=.5em @!R{
\freegate{\bra{j}}
&\gate{H}
&\timesgate\qwx[1]
&\gate{\ket{k}}
&\freegate{\bra{k}}
&\timesgate\qwx[1]
&\qw
&\gate{\ket{\psi}}
\\
&\qw
&\dotgate
&\qw
&\qw
&\dotgate
&\gate{H}
&\gate{\ket{0}}
\gategroup{1}{1}{2}{4}{.7em}{.}
\gategroup{1}{5}{2}{8}{.7em}{.}
}
\end{array}
\label{eq-scat-two-boxes}
\\
&=&
\left[
\frac{(-1)^{jk}}{\sqrt{2}}
\sigz^j(\bitb)
\right]
\left[
\frac{\sigx^k(\bitb)}{\sqrt{2}}\ket{\psi}_\bitb
\right]
\label{eq-scat-no-boxes}
\\
&=&
\sigx^k(\bitb)\sigz^j(\bitb)
\ket{\psi}_\bitb/2
\;.
\eeqa
To go from Eq.(\ref{eq-scat-one-box})
to Eq.(\ref{eq-scat-two-boxes}), we
expressed the two qubit projective measurement
in terms of 2 CNOTs, as described in
the section entitled One and Two Qubit
Projective Measurements.
To go from Eq.(\ref{eq-scat-two-boxes})
to Eq.(\ref{eq-scat-no-boxes}), we used identity
Eq.(\ref{eq-ex-scat-via-cnot2})
to reduce the second dotted
box of Eq.(\ref{eq-scat-two-boxes}).
\altproof
For any $a\in Bool$,
\beqa
\lefteqn{
\begin{array}{c}
\Qcircuit @C=1em @R=.5em @!R{
\freegate{\bra{j}}
&\gate{H}
&\multigate{1}{\pizz{k}}
&\qw
&\gate{\ket{a}}
\\
&\qw
&\ghost{\pizz{k}}
&\gate{H}
&\gate{\ket{0}}
}
\end{array}
=}\\
&=&
\bra{j}_\bita
H(\bita)
\pizz{k}
\left(\sum_{(a',b')\in Bool^2}
\ket{a',b'}
\bra{a',b'}\right)
H(\bitb)
\ket{a,0}_{\bita\bitb}\\
&=&
\sum_{a',b'}
\bra{j}_\bita
H(\bita)
\ket{a'}_\bita
\ket{b'}_\bitb
\delta^k_{a'\oplus b'}
\bra{a',b'}H(\bitb)
\ket{a, 0}_{\bita\bitb}\\
&=&
\sum_{a',b'}
\frac{(-1)^{ja'}}{\sqrt{2}}
\ket{b'}_\bitb
\delta^k_{a'\oplus b'}
\frac{\delta_a^{a'}}{\sqrt{2}}\\
&=&
\frac{(-1)^{ja}}{2}
\ket{k\oplus a}_\bitb\\
&=&
\sigx^k(\bitb)\sigz^j(\bitb)
\ket{a}_\bitb/2
\;.
\eeqa
\qed
