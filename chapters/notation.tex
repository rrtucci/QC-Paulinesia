
\chapter{Notation}


Let $Bool = \{0,1\}$.
For integers $a$ and $b$ such that $a\leq b$,
let $Z_{a,b} = \{a, a+1, a+2, \ldots b\}$.


$\delta(x, y)$ and $\delta^x_y$ will both
denote the Kronecker delta function.
It equals one when $x=y$ and zero otherwise.

For any statement ${\cal S}$,
we define the truth function $\theta({\cal S})$
to equal 1 if ${\cal S}$ is true
and 0 if ${\cal S}$ is false.
For example, $\theta(x>0)$ represents the unit step
function and $\delta(x, y)=\theta(x=y)$
the Kronecker delta function.


$\oplus$ will denote
addition mod 2.
Hence, for any $a,b\in Bool$,
$a\oplus b = a + b - 2ab$ and
$(-1)^{a\oplus b} = (-1)^{a+b}$.
When speaking of bits with states
0 and 1, we will often use
an overline to represent
the opposite state:
$\bar{0} = 1$, $\bar{1} = 0$.
Note that if $x, k\in Bool$, then
$\sum_k (-1)^{kx} = 1 + (-1)^x = 2 \delta(x, 0)$.
For $x\in Bool$, $\delta(x,1)=x$.

We will often use $\ns=2^\nb$,
where $\nb$ stands for number of bits and
$\ns$ for number of states.
We will use lower case Latin letters
$a,b,c\ldots \in Bool$ to
represent bit values and
lower case Greek letters
$\alpha, \beta, \gamma, \ldots \in Z_{0, \nb-1} $
to represent bit positions.

Given a binary vector
$\vec{x}\in Bool^\nb$,
if its components are
labelled as follows:
$\vec{x} =
(x_{\nb-1}, x_{\nb-2}, \ldots, x_1, x_0)$,
then
we will say that the components of
$\vec{x}$ are labelled naturally.
For some applications, it is very convenient
to use natural labelling. For other applications,
it doesn't much matter whether we use
natural labelling or not.
In cases where it doesn't matter, we may
use other common labellings such as
$\vec{x} =
(x_1, x_2, \ldots, x_\nb)$.

Let
$\vec{\nu}=
(\nb-1, \nb-2, \ldots, 1,0)$, and
$2^{\vec{\nu}}=(2^{\nb-1}, 2^{\nb-2}, \ldots, 2^1,2^0)$.

Given any $x\in Z_{0, \ns-1}$, we can write
$x=\sum_{i=0}^{\nb-1}2^i x_i$.
If we define the naturally labelled
binary vector $\vec{x} =
(x_{\nb-1}, \ldots, x_1, x_0)$,
then $x=2^{\vec{\nu}}\cdot \vec{x}$.
We call $\vec{x} =
(x_{\nb-1}, \ldots, x_1, x_0)$
 the binary representation
of $x$ and denote it by $bin(x)$.

Given any naturally labelled
binary vector $\vec{x} =
(x_{\nb-1}, \ldots, x_1, x_0)$,
we can write $x=2^{\vec{\nu}}\cdot \vec{x}$.
We call $x\in Z_{0, \ns-1}$ the decimal representation
of $\vec{x}$ and denote it by $dec(\vec{x})$.

If $\vec{x},\vec{y}\in Bool^\nb$,
we will use $\vec{x}\cdot \vec{y}=
 \sum_{i=0}^{\nb-1} x_i y_i$,
 where the addition is normal, not mod 2.

We define the single-qubit states
$\ket{0}$ and $\ket{1}$ by

\beq
\ket{0} =
\left[
\begin{array}{c}
1 \\ 0
\end{array}
\right]
\;\;,\;\;
\ket{1} =
\left[
\begin{array}{c}
0 \\ 1
\end{array}
\right]
\;.
\eeq
Given any
$\vec{x}=(x_1, x_2, \ldots, x_\nb) \in Bool^{\nb}$,
and given a vector of distinct qubit labels
$\vecbitb = (\bitb_1, \bitb_2,
 \ldots, \bitb_\nb)$,
we define the
$\nb$-qubit state
$\ket{\vec{x}}$ as the following tensor product

\beq
\ket{\vec{x}} =
\ket{\vec{x}}_\vecbitb =
\ket{x_1}_{\bitb_1}
\ket{x_2}_{\bitb_2}
\ldots
\ket{x_\nb}_{\bitb_\nb}=
\ket{x_1}\otimes
\ket{x_2}
\ldots\otimes
\ket{x_\nb}
\;.
\label{eq-ket-vec-x-unnatural}
\eeq
For example,

\beq
\ket{01} =
\left[
\begin{array}{c}
1 \\ 0
\end{array}
\right]
\otimes
\left[
\begin{array}{c}
0 \\ 1
\end{array}
\right]
=
\left[
\begin{array}{c}
0 \\ 1 \\ 0 \\0
\end{array}
\right]
\;.
\eeq
With natural labelling,
we would use $\vec{x} =
(x_{\nb-1}, \ldots, x_1, x_0)$,
$\vecbitb=\vec{\nu}$
and $x=\sum_{i=0}^{\nb-1}2^i x_i$.
Instead of
Eq.(\ref{eq-ket-vec-x-unnatural}),
we would have

\beq
\ket{x}=
\ket{\vec{x}} =
\ket{\vec{x}}_{\vec{\nu}} =
\ket{x_{\nb-1}}_{\nb-1}
\ldots
\ket{x_1}_{1}
\ket{x_0}_{0}
=
\ket{x_{\nb-1}}\otimes
\ldots\otimes
\ket{x_1}\otimes
\ket{x_0}
\;.
\label{eq-ket-vec-x-natural}
\eeq

Of course, any $\nb$ qubit state
can be obtained as a linear combination of
the states $\ket{\vec{x}}$
for all $\vec{x}\in Bool^\nb$.

$I_r$ will represent the $r$ dimensional
unit matrix, for any integer $r\geq 1$.

Suppose
$\vecbitb =(\bitb_1, \bitb_2, \ldots, \bitb_\nb)$
is a vector of bit labels,
$(M_1, M_2, \ldots , M_\nb)$ is
a vector of  $2\times 2$ complex matrices,
and
$(\phi_1, \phi_2, \ldots , \phi_\nb)$ is
a vector of  2-dimensional complex column vectors.
For $i\in Z_{1, \nb}$, we define
$M_i(\beta_i)$ by

\beq
M_i(\beta_i) =
I_2 \otimes
\cdots \otimes
I_2 \otimes
M_i \otimes
I_2 \otimes
\cdots \otimes
I_2
\;,
\eeq
where the matrix $M_i$ on the right
hand side is located
at bit position $i$
(counting from left to right, starting at 1)
 in the tensor product
of $\nb$ $2\times 2$ matrices.
We often define a product operator
$M(\vecbitb)$ by

\beq
M(\vecbitb) = \prod_{i=1}^\nb M_i(\beta_i)=
M_1(\bitb_1)\otimes
M_2(\bitb_2)\otimes
\ldots
M_\nb(\bitb_\nb)
\;,
\eeq
and a product state $\ket{\phi}_\vecbitb$

\beq
\ket{\phi}_\vecbitb =
\prod_{i=1}^\nb \ket{\phi_i}_{\beta_i}=
\ket{\phi_1}\otimes
\ket{\phi_2}\otimes
\ldots
\ket{\phi_\nb}
\;.
\eeq
For example, we might find it useful to define
an operator $M(\vecbitb)$
and a state
$\ket{\phi}_{\vecbitb}$
 by

 \beq
M(\vecbitb)
=
\prod_{i=1}^{\nb}\sigx(\bitb_i)=
\sigx\otimes
\sigx\otimes
\ldots\otimes
\sigx
\;,
\eeq

\beq
\ket{\phi}_{\vecbitb}=
\ket{0}_{\vecbitb} =
\prod_{i=1}^{\nb}\ket{0}_{\bitb_i}=
\left(\begin{array}{c}1\\0\end{array}\right)
\otimes
\left(\begin{array}{c}1\\0\end{array}\right)
\otimes\cdots\otimes
\left(\begin{array}{c}1\\0\end{array}\right)
=
[1,0,0,\ldots,0]^T
\;.
\eeq



With natural labelling, we use
$\vecbitb = \vec{\nu}$.
Let
$(M_{\nb-1}, \ldots , M_1, M_0)$
be a vector of
 $2\times 2$ complex matrices,
and let
$(\phi_{\nb-1}, \ldots , \phi_1, \phi_0)$
be a vector of
 2-dimensional complex column vectors.
 With natural labelling,
for $i\in Z_{0, \nb-1}$, we define
$M_i(i)$ by

\beq
M_i(i) =
I_2 \otimes
\cdots \otimes
I_2 \otimes
M_i \otimes
I_2 \otimes
\cdots \otimes
I_2
\;,
\eeq
where the matrix $M_i$ on the right
hand side is located
at bit position $i$
(counting from right to left, starting at 0)
 in the tensor product
of $\nb$ $2\times 2$ matrices.
We often define a product operator
$M(\vec{\nu})$ by

\beq
M(\vec{\nu}) = \prod_{i=0}^{\nb-1} M_i(i)=
M_{\nb-1}(\nb-1)\otimes
\ldots\otimes
M_1(1)\otimes
M_0(0)
\;,
\eeq
and a product state $\ket{\phi}_{\vec{\nu}}$

\beq
\ket{\phi}_{\vec{\nu}} =
\prod_{i=0}^{\nb-1} \ket{\phi_i}_{i}=
\ket{\phi_{\nb-1}}\otimes
\ldots\otimes
\ket{\phi_1}\otimes
\ket{\phi_0}
\;.
\eeq


Next we explain
our circuit diagram notation.
In our
qubit circuit diagrams,
each horizontal wire represents
a single qubit (except when
stated explicitly that the
wire represents several qubits). Different
wires  represent different qubits.
We label single qubit wires
by Greek letters or
by integers as follows:

\beq
\begin{array}{c}
\Qcircuit @C=1em @R=1em @!R{
&\rstick{\bita}\qw
\\
&\rstick{\bitb}\qw
\\
&\rstick{\bitc}\qw
\\
&\rstick{\bitd}\qw
\\
&\vdots
}
\end{array}
\;\;\;\;\;\;,\;\;\;
\begin{array}{c}
\Qcircuit @C=1em @R=1em @!R{
&\rstick{0}\qw
\\
&\rstick{1}\qw
\\
&\rstick{2}\qw
\\
&\rstick{3}\qw
\\
&\vdots
}
\end{array}
\;\;\;\;\;.
\label{eq-qwires}
\eeq
Thus, the first (topmost) wire
is labelled either
$\alpha$ or $0$, the second wire
is labelled either $\beta$ or $1$,
and so forth.
For some special applications,
we label qubits differently from
Eq.(\ref{eq-qwires}). For example,
we might label the first two wires
$\bita_1, \bita_2$, and the next
two wires $\bitb_1, \bitb_2$,
or we might want to label the first
wire $(\bita_1, \bita_2)$,
and make it represent two qubits.
In cases where bit labelling is different
from Eq.(\ref{eq-qwires}),
this will be stated explicitly.
Bras are represented by
\beq
\ket{\psi_1}_\bita
\ket{\psi_2}_\bitb
=
\begin{array}{c}
\Qcircuit @C=1em @R=.5em @!R{
&\gate{\ket{\psi_1}}
\\
&\gate{\ket{\psi_2}}
}
\end{array}
\;\;,\;\;
\ket{\psi}_{\bita\bitb}
=
\begin{array}{c}
\Qcircuit @C=1em @R=1.5em @!R{
&\multigate{1}{\ket{\psi}}
\\
&\ghost{\ket{\psi}}
}
\end{array}
\;,
\eeq
and kets by

\beq
\bra{\chi_1}_\bita
\bra{\chi_2}_\bitb
=
\begin{array}{c}
\Qcircuit @C=1em @R=.5em @!R{
&\freegate{\bra{\chi_1}}&\qw
\\
&\freegate{\bra{\chi_2}}&\qw
}
\end{array}
\;\;,\;\;
\bra{\chi}_{\bita\bitb}
=
\begin{array}{c}
\Qcircuit @C=1em @R=1.5em @!R{
&\freemultigate{1}{\bra{\chi}}&\qw
\\
&\freeghost{\bra{\chi}}&\qw
}
\end{array}
\;.
\eeq
Operators are represented by
\beq
T_1(\bita) T_2(\bitb)
=
\begin{array}{c}
\Qcircuit @C=1em @R=.5em @!R{
&\gate{T_1}&\qw
\\
&\gate{T_2}&\qw
}
\end{array}
\;\;,\;\;
T(\bita,\bitb)
=
\begin{array}{c}
\Qcircuit @C=1em @R=1.5em @!R{
&\multigate{1}{T}&\qw
\\
&\ghost{T}&\qw
}
\end{array}
\;.
\eeq
Matrix elements are represented
by combining the above rules
for bras, kets, and operators.
For example,

\beq
\bra{\chi}_{\bita\bitb}
T(\bita,\bitb)\ket{\psi}_{\bita\bitb}
=
\begin{array}{c}
\Qcircuit @C=1em @R=1.5em @!R{
\freemultigate{1}{\bra{\chi}}
&\multigate{1}{T}
&\multigate{1}{\ket{\psi}}
\\
\freeghost{\bra{\chi}}
&\ghost{T}
&\ghost{\ket{\psi}}
}
\end{array}
\;.
\eeq
{\it Note that in our circuit diagrams,
time flows from the right to the left
of the diagram.} Careful:
Many workers in Quantum
Computing draw their diagrams
so that time flows from
left to right. We eschew their
convention because
it forces one to reverse
the order of the operators
every time one wishes to convert
between a circuit
diagram
and its algebraic equivalent
in Dirac Notation.

Next, we will introduce
a slight enhancement to the standard Dirac
Notation. Given a
ket $\ket{\psi}$,
if we can find an operator
$\Omega$ such that
$\ket{\psi}$ is
a unique (up to a scalar factor) eigenvector
of $\Omega$ with eigenvalue $\lambda$,
then we will sometimes denote
$\ket{\psi}$ by
$\ket{\Omega=\lambda}$.
Sometimes, in order to
specify $\ket{\psi}$ uniquely,
one needs to
find a complete set of commuting operators
$\{\Omega_i : i\in Z_{1, N}\}$
such that $\Omega_i\ket{\psi} =
\lambda_i\ket{\psi}$
for all $i$,
and then we can denote $\ket{\psi}$
by
$\ket{\vec{\Omega}=\vec{\lambda}}$.
Note that if $U$ is a unitary
operator that acts on the same
Hilbert space as an operator $\Omega$, then
$\ket{U\Omega U^\dagger=\lambda}=
U\ket{\Omega=\lambda}$.
If
operator
$\Omega$
has an
eigenspace with eigenvalue $\lambda$,
then we denote the projector onto that
eigenspace by
$\pi(\Omega=\lambda)$.
If the eigenspace is one dimensional, then
$\pi(\Omega=\lambda)=\ket{\Omega=\lambda}
\bra{\Omega=\lambda}$.
If the eigenspace has dimension greater than one,
then we can always find an
orthonormal  basis
$\{\ket{\psi_\lambda^i}: i\in S\}$
for the eigenspace,
and then
$\pi(\Omega=\lambda)=
\sum_{i\in S}
\ket{\psi_\lambda^i}
\bra{\psi_\lambda^i}$.
Note that if $U$ is a unitary
operator
that acts on the same
Hilbert space as operator $\Omega$,
then
$U\pi(\Omega=\lambda)U^\dagger=
\pi(U\Omega U^\dagger=\lambda)$.

The Pauli matrices are defined by:

\beq
\sigx =
\left(
\begin{array}{cc}
0&1\\
1&0
\end{array}
\right)
\;,\;\;
\sigy =
\left(
\begin{array}{cc}
0&-i\\
i&0
\end{array}
\right)\;,\;\;
\sigz =
\left(
\begin{array}{cc}
1&0\\
0&-1
\end{array}
\right)
\;.
\eeq
More information about the Pauli matrices
may be found in the section
entitled Pauli Matrices.

We will often abbreviate
$n-$fold tensor
products of Pauli matrices
as follows. If $w_1, w_2,
\ldots, w_n\in \{X,Y, Z\}$, and
$b_1, b_2,
\ldots, b_n\in Bool$, then let

\beq
\sigma_{w_1,w_2, \ldots, w_n}^{b_1, b_2, \ldots, b_n} =
\sigma_{w_1}^{b_1}\otimes \sigma_{w_2}^{b_2}
\otimes\ldots
\otimes \sigma_{w_n}^{b_n}
\;.
\eeq
For example,
$\sigma_{XYY}^{1,0,1}=
\sigx^1\otimes \sigy^0 \otimes \sigy^1$.
Equivalently, for $n$ bits
$\vecbita=(\alpha_1, \alpha_2,\ldots \alpha_n)$,

\beq
\sigma_{w_1,w_2, \ldots, w_n}^{b_1, b_2, \ldots, b_n}
(\vec{\alpha})=
\prod_{i=1}^n
\sigma_{w_i}^{b_i}(\bita_i)
\;.
\eeq
Also let

\beq
\sigma_{w_1,w_2, \ldots, w_n}=
\sigma_{w_1,w_2, \ldots, w_n}^{1,1, \ldots, 1}=
\sigma_{w_1}\otimes \sigma_{w_2}
\otimes\ldots
\otimes \sigma_{w_n}
\;.
\eeq
For example,
$\sigma_{XYY}=\sigma_{XYY}^{1,1,1}=
\sigx\otimes \sigy \otimes \sigy$.

It is sometimes convenient to define
the following operator for any $x,z\in Bool$
and any qubit $\alpha$:

\beq
\Lambda^{x, z}(\bita)=
\sigx^{x}(\bita)
\sigz^{z}(\bita)
\;.
\eeq
Note that
$\Lambda^{x,z\dagger}=(-1)^{xz}\Lambda^{x, z}$,
and
$\Lambda^{00}=1$,
$\Lambda^{10}=\sigx$,
$\Lambda^{11}=(-i)\sigy$,
$\Lambda^{00}=\sigz$.
$\Lambda^{x, z}$ arises, for example,
when dealing with Bell states.


For any $j\in Bool$ and
$w_1, w_2\in \{X,Y, Z\}$, let
$\Pi_{w_1, w_2}^{j}$ be the projection operator
that projects the
2 qubit Hilbert space onto
the eigenspace of $\sigma_{w_1,w_2}$
with eigenvalue
$(-1)^j$. Thus,

\beq
\Pi_{w_1, w_2}^{j} =
\pi[\sigma_{w_1,w_2}=(-1)^j]
\;.
\eeq
Note that
\beq
\sigma_{ZZ}=
\sigz\otimes \sigz=
\left(
\begin{array}{cc}
1&0\\
0&-1
\end{array}
\right)
\otimes
\left(
\begin{array}{cc}
1&0\\
0&-1
\end{array}
\right)
=
diag(1,-1,-1,1)
\;.
\label{eq-sigzz}
\eeq
From Eq.(\ref{eq-sigzz}),
it is clear that
for any $j,a,b \in Bool$,

\beq
\pizz{j}\ket{a,b}
=\delta^j_{a\oplus b}\ket{a,b}
\;.
\eeq
