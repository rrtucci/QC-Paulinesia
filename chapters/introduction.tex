\chapter{Introduction}

This document is not
a  full course in Quantum Computing.
My goal in producing it
was to create a collection of
qubit circuit identities
that are used
in Quantum Computing.
Mathematicians and
Physicists may consider it
 as being analogous to a
Table of Integrals  or a
 Mathematical Handbook
such as Gradshteyn \& Ryzhik
or Abramowitz \& Stegun.
Computer Programmers may think of it
as a scrapbook of code snippets
that are elegant, instructive,
well documented, and useful.
Electronics experts
may view it as a compendium of
circuits for performing
a large assortment of
tasks.

The vast majority of the  circuit
identities collected in this work
were not discovered for the first time by me, and
I take no credit for discovering them.
In producing this document, I am
acting as a collector,
not as a discoverer.

I plan to
continue adding
qubit circuit identities to this
collection, and to release future versions
of this document
containing the new specimens.
For example, there are some
nice identities involving
quantum error correction and quantum
compiling that I have not included yet,
but which I plan to include in future versions.
Suggestions and comments are
welcomed and appreciated.

This document benefitted greatly
from the wonderful LaTeX
macros: QCircuit (by
B. Eastin, S. T. Flammia)
and XYPic (by K.H. Rose and R.R. Moore),
on which QCircuit is based.
