\chapter{Exchanger}

We define the Exchanger
(a.k.a. Swapper or Exchange Operator
or Bit Transposition) by

\grayeq{
\beq
E(\bita, \bitb)
\ket{a, b}_{\bita\bitb}
=
\ket{b, a}_{\bita\bitb}
\;,
\eeq}
for all $a, b \in Bool$.
Therefore

\grayeq{
\beq
E(\bita, \bitb) = E(\bitb, \bita)
\;,
\eeq}
and

\grayeq{
\beq
E(\bita, \bitb)^2 = 1
\;.
\eeq}
Throughout QC Paulinesia,
we will represent Exchanger by

\grayeq{
\beq
E(\bita, \bitb)
=
\begin{array}{c}
\Qcircuit @C=1em @R=1em @!R{
&\uarrowgate\qwx[1]
&\qw
\\
&\darrowgate
&\qw
}
\end{array}
\;.
\eeq}

\claim

\grayeq{
\beq
E(\bita, \bitb)=
\cnot{\bitb}{\bita}
\cnot{\bita}{\bitb}
\cnot{\bitb}{\bita}
=
\begin{array}{c}
\Qcircuit @C=1em @R=1em @!R{
&\timesgate\qwx[1]
& \dotgate\qwx[1]
& \timesgate\qwx[1]
&\qw \\
&\dotgate
& \timesgate
& \dotgate
& \qw
}
\end{array}
\;.
\eeq}
\proof
\beqa
\cnot{\bitb}{\bita}
\cnot{\bita}{\bitb}
\cnot{\bitb}{\bita}
\ket{a,b}_{\bita\bitb}&=&
\cnot{\bitb}{\bita}
\cnot{\bita}{\bitb}
\ket{a \oplus b,b}\\
&=&
\cnot{\bitb}{\bita}
\ket{a \oplus b,a}\\
&=&
\ket{b,a}
\;.
\eeqa
\qed

\claim
If  $U$ and $V$
are $2\times2$ unitary matrices, then

\grayeq{
\beq
\begin{array}{c}
\Qcircuit @C=1em @R=1em @!R{
&\gate{U}
&\uarrowgate\qwx[1]
&\gate{V^\dagger}
&\qw
\\
&\gate{V}
&\darrowgate
&\gate{U^\dagger}
&\qw
}
\end{array}
=
\begin{array}{c}
\Qcircuit @C=1em @R=1em @!R{
&\uarrowgate\qwx[1]
&\qw
\\
&\darrowgate
&\qw
}
\end{array}
\;.
\label{eq-exc-two-u-invar}
\eeq}
\proof
Obvious.
\qed

\claim

\grayeq{
\beq
\begin{array}{c}
\Qcircuit @C=1em @R=1em @!R{
&\timesgate\qwx[1]
& \dotgate\qwx[1]
& \timesgate\qwx[1]
&\qw
\\
&\dotgate
& \timesgate
& \dotgate
&\qw
}
\end{array}
=
\begin{array}{c}
\Qcircuit @C=1em @R=1em @!R{
&\timesgate\qwx[1]
& \ogate\qwx[1]
& \timesgate\qwx[1]
&\qw
 \\
&\ogate
& \timesgate
& \ogate
&\qw
}
\end{array}
=
\begin{array}{c}
\Qcircuit @C=1em @R=1em @!R{
&\ogate
& \timesgate
& \ogate
&\qw
 \\
&\timesgate\qwx[-1]
& \ogate\qwx[-1]
& \timesgate\qwx[-1]
&\qw
}
\end{array}
=
\begin{array}{c}
\Qcircuit @C=1em @R=1em @!R{
&\dotgate
& \timesgate
& \dotgate
&\qw
 \\
&\timesgate\qwx[-1]
& \dotgate\qwx[-1]
& \timesgate\qwx[-1]
&\qw
}
\end{array}
\;.
\eeq
}
\proof
By virtue of Eq.(\ref{eq-exc-two-u-invar}),

\beq
\begin{array}{c}
\Qcircuit @C=1em @R=1em @!R{
&\timesgate\qwx[1]
& \dotgate\qwx[1]
& \timesgate\qwx[1]
&\qw
\\
&\dotgate
& \timesgate
& \dotgate
&\qw
}
\end{array}
=
\begin{array}{c}
\Qcircuit @C=1em @R=1em @!R{
&\gate{\sigx}
&\timesgate\qwx[1]
& \dotgate\qwx[1]
& \timesgate\qwx[1]
&\gate{\sigx}
&\qw
 \\
&\gate{\sigx}
&\dotgate
& \timesgate
& \dotgate
&\gate{\sigx}
&\qw
}
\end{array}
=
\begin{array}{c}
\Qcircuit @C=1em @R=1em @!R{
&\timesgate\qwx[1]
& \ogate\qwx[1]
& \timesgate\qwx[1]
&\qw
\\
&\ogate
& \timesgate
& \ogate
&\qw
}
\end{array}
\;.
\eeq
Likewise,

\beq
\begin{array}{c}
\Qcircuit @C=1em @R=1em @!R{
&\timesgate\qwx[1]
& \dotgate\qwx[1]
& \timesgate\qwx[1]
&\qw
\\
&\dotgate
& \timesgate
& \dotgate
&\qw
}
\end{array}
=
\begin{array}{c}
\Qcircuit @C=1em @R=1em @!R{
&\gate{H}
&\timesgate\qwx[1]
& \dotgate\qwx[1]
& \timesgate\qwx[1]
&\gate{H}
&\qw
 \\
&\gate{H}
&\dotgate
& \timesgate
& \dotgate
&\gate{H}
&\qw
}
\end{array}
=
\begin{array}{c}
\Qcircuit @C=1em @R=1em @!R{
&\dotgate
& \timesgate
& \dotgate
&\qw
 \\
&\timesgate\qwx[-1]
& \dotgate\qwx[-1]
& \timesgate\qwx[-1]
&\qw
}
\end{array}
\;.
\eeq
\qed


\claim

\grayeq{
\beq
E(\bita, \bitb)
=
[n(\bita)n(\bitb) + \nbar(\bita)\nbar(\bitb)]
+
\sigx(\bita)\sigx(\bitb)
[n(\bita)\nbar(\bitb) + \nbar(\bita)n(\bitb)]
\;.
\label{eq-exc-n-nbar}
\eeq}
\proof
Let RHS be the right hand side
of Eq.(\ref{eq-exc-n-nbar}).
For any $a,b\in Bool$,
if $a=b$, $RHS\ket{a,b}=\ket{a,b}$,
whereas when $a\neq b$,
$RHS\ket{a,b}=\ket{\overline{a},\overline{b}}$.
\qed

For any $x,z\in Bool$ and bit $\bita$, let
$\Lambda^{x,z}(\bita) =
\sigx^x(\bita)
\sigz^z(\bita)$. Note that
$[\Lambda^{x,z}]^\dagger
 = (-1)^{xz}\Lambda^{x,z}$
 and that
 $\Lambda^{00} = 1$,
 $\Lambda^{10} = \sigx$,
 $\Lambda^{11} = (-i)\sigy$,
 $\Lambda^{01} = \sigz$.
 As usual, let
 $\sigma_{w_1 w_2} =
 \sigma_{w_1}\otimes\sigma_{w_2}$
 for $w_1, w_2 \in \{X,Y,Z\}$.

\claim

\grayeq{
\beqa
E(\bita, \bitb)
&=&
\frac{1}{2}
\sum_{(x,z)\in Bool^2}
\Lambda^{xz}(\bita)
[\Lambda^{xz}(\bitb)]^\dagger\\
&=&
\frac{1}{2}
( 1
+ \sigma_{XX}
+ \sigma_{YY}
+ \sigma_{ZZ})
(\bita, \bitb)\\
&=&
\frac{1}{2}[
1 + \vec{\sigma}(\bita)
\cdot \vec{\sigma}(\bitb)]
\label{eq-exc-dot-prod}
\;.
\eeqa}
\proof
\beqa
\lefteqn{\frac{1}{2}
\sum_{x,z}
\Lambda^{xz}(\bita)(-1)^{xz}
\Lambda^{xz}(\bitb)
\ket{a,b}_{\bita, \bitb} =}\\
&=&
\frac{1}{2}
\sum_{x,z}
(-1)^{xz}
(\sigx^x\sigz^z \ket{a}_\bita)
(\sigx^x\sigz^z \ket{b}_\bitb)\\
&=&
\frac{1}{2}
\sum_{x,z}
(-1)^{(x+a+b)z}
\ket{a\oplus x, b \oplus x}\\
&=&
\frac{1}{2}
\sum_x 2 \delta^x_{a\oplus b}
\ket{a\oplus x, b \oplus x}\\
&=&
\ket{b,a}
\;.
\eeqa
\altproof
Replace the 3 CNOTs
in
$E(\bita, \bitb) =
\cnot{\bitb}{\bita}
\cnot{\bita}{\bitb}
\cnot{\bitb}{\bita}$
by
$\cnot{\bitb}{\bita}=
\frac{1}{2} \sum_{x,z} \sigx^x(\bita)
\sigz^z(\bitb)(-1)^{xz}$.
Details left to the reader.
\qed

We could
have predicted that
 $E(\bita, \bitb)$ would have the form
 Eq.(\ref{eq-exc-dot-prod})
 due to the invariance of Exchanger
 under identical rotations of both bits;
 that is, due to
 Eq.(\ref{eq-exc-two-u-invar})
 with
 $U=V=e^{i\vec{\theta}
 \cdot \vec{\sigma}}$,
 where $\vec{\theta}$ is
 an arbitrary
 3 dimensional real vector.

 \claim

 \grayeq{
\beq
\begin{array}{c}
\Qcircuit @C=1em @R=1em @!R{
&\uarrowgate\qwx[2]
&\qw
\\
&\qw
&\qw
\\
&\darrowgate
&\qw
}
\end{array}
=
\begin{array}{c}
\Qcircuit @C=1em @R=1em @!R{
&\uarrowgate\qwx[1]
&\qw
&\uarrowgate\qwx[1]
&\qw
\\
&\darrowgate
&\uarrowgate\qwx[1]
&\darrowgate
&\qw
\\
&\qw
&\darrowgate
&\qw
&\qw
}
\end{array}
\;.
\eeq}
\proof
Check that both sides map
$\bita\rarrow\bitc$,
$\bitb\rarrow\bitb$,
$\bitc\rarrow\bita$.
\qed
