\chapter{CNOTs}


We define a CNOT
(C = controlled, NOT $=\sigx$) by:

\grayeq{
\beq
CNOT(\bita\rarrow\bitb)=
CNOT(\bitb\larrow\bita)=
\cnot{\bita}{\bitb}=
(-1)^{n(\bita)n_X(\bitb)}=
\begin{array}{c}
\Qcircuit @C=1em @R=1em @!R{
&\dotgate\qwx[1]
&\qw
\\
&\timesgate
&\qw
}
\end{array}
\;.
\eeq}
$\bita$ is called the
{\bf control qubit} and $\bitb$ is called the
{\bf target qubit}. The CNOT can be easily
generalized to have more than one control qubit:

\grayeq{
\beq
\sigma_X^{n(\bita)n(\bitb)}(\bitc)=
(-1)^{n(\bita)n(\bitb)n_X(\bitc)}=
\begin{array}{c}
\Qcircuit @C=1em @R=1em @!R{
&\dotgate\qwx[2]
&\qw
\\
&\dotgate
&\qw
\\
&\timesgate
&\qw
}
\end{array}
\;.
\eeq}
Other operators related to CNOT are

\grayeq{
\beq
\cnoto{\bita}{\bitb}=
(-1)^{\nbar(\bita)n_X(\bitb)}=
\begin{array}{c}
\Qcircuit @C=1em @R=1em @!R{
&\ogate\qwx[1]
&\qw
\\
&\timesgate
&\qw
}
\end{array}
\;,
\eeq}
and

\grayeq{
\beq
\sigz^{n(\bita)}(\bitb)=
\sigz^{n(\bitb)}(\bita)=
(-1)^{n(\bita)n(\bitb)}=
\begin{array}{c}
\Qcircuit @C=1em @R=1em @!R{
&\dotgate\qwx[1]
&\qw
\\
&\dotgate
&\qw
}
\end{array}
\;.
\eeq}
For any $a,b,c\in Bool$,

\grayeq{
\beq
\cnot{\bita}{\bitb}\ket{a,b}_{\bita\bitb}=
\ket{a, b\oplus a}
\;,
\label{eq-cnot-oplus}
\eeq}

\grayeq{
\beq
\sigma_X^{n(\bita)n(\bitb)}(\bitc)
\ket{a,b,c}_{\bita\bitb\bitc}=
\ket{a, b, c\oplus ab}
\;,
\eeq}

\grayeq{
\beq
\cnoto{\bita}{\bitb}\ket{a,b}_{\bita\bitb}=
\ket{a, b \oplus\overline{a} }
\;,
\eeq}

\grayeq{
\beq
(-1)^{n(\bita)n(\bitb)}\ket{a,b}_{\bita\bitb}=
(-1)^{ab}\ket{a, b}
\;.
\eeq}

Some workers represent a CNOT by
$\begin{array}{c}
\Qcircuit @C=1em @R=.5em @!R{
&\dotgate\qwx[1]
&\qw
\\
&\targ
&\qw
}
\end{array}$
instead of
$\begin{array}{c}
\Qcircuit @C=1em @R=1em @!R{
&\dotgate\qwx[1]
&\qw
\\
&\timesgate
&\qw
}
\end{array}$.
The
$\begin{array}{c}
\Qcircuit @C=1em @R=.5em @!R{
&\dotgate\qwx[1]
&\qw
\\
&\targ
&\qw
}
\end{array}$
notation reminds us of the $\oplus$
in Eq.(\ref{eq-cnot-oplus}), whereas the
$\begin{array}{c}
\Qcircuit @C=1em @R=1em @!R{
&\dotgate\qwx[1]
&\qw
\\
&\timesgate
&\qw
}
\end{array}$
notation reminds us
of the $X$ in $\cnot{\bita}{\bitb}$.


\claim

\grayeq{
\beq
\cnot{\bitb}{\bita}=
\sigx(\bita)n(\bitb) + \nbar(\bitb)
\;.
\eeq}
\proof
Check that both sides agree when $n(\bitb)$
equals zero and one.
\qed

\claim

\grayeq{
\beq
\cnot{\bitb}{\bita}=
\frac{1}{2}\sum_{(x,z)\in Bool^2}
\sigx^x(\bita)\sigz^z(\bitb) (-1)^{xz}
\;.
\eeq}
\proof
\beqa
\cnot{\bitb}{\bita} &=&
(-1)^{n_X(\bita)n_Z(\bitb)}\\
&=& 1 - 2 n_X(\bita)n_Z(\bitb)\\
&=& 1 - 2
\left(\frac{1-\sigx(\bita)}{2}\right)
\left(\frac{1-\sigz(\bitb)}{2}\right)\\
&=&
\frac{1}{2}
[ 1 + \sigx(\bita) + \sigz(\bitb)
- \sigma_{XZ}(\bita, \bitb)]
\;.
\eeqa
\qed


\claim (Permuting 2 CNOTs in a chain)

\grayeq{
\beqa
\begin{array}{c}
\Qcircuit @C=1em @R=1em @!R{
&\timesgate\qwx[1]
&\qw
&\qw
\\
&\dotgate
&\timesgate\qwx[1]
&\qw
\\
&\qw
&\dotgate
&\qw
}
\end{array}
&=&
\begin{array}{c}
\Qcircuit @C=1em @R=1em @!R{
&\timesgate\qwx[2]
&\qw
&\timesgate\qwx[1]
&\qw
\\
&\qw
&\timesgate\qwx[1]
&\dotgate
&\qw
\\
&\dotgate
&\dotgate
&\qw
&\qw
\gategroup{1}{2}{3}{2}{.7em}{.}
}
\end{array}
\label{eq-com-cnots-line}
\\
&=&
\begin{array}{c}
\Qcircuit @C=1em @R=1em @!R{
&\qw
&\timesgate\qwx[1]
&\timesgate\qwx[2]
&\qw
\\
&\timesgate\qwx[1]
&\dotgate
&\qw
&\qw
\\
&\dotgate
&\qw
&\dotgate
&\qw
\gategroup{1}{4}{3}{4}{.7em}{.}
}
\end{array}
\;.
\eeqa}
\proof
Let LHS and RHS stand for the left and
right hand sides of Eq.(\ref{eq-com-cnots-line}).
For $a,b,c\in Bool$,

\beqa
LHS\ket{a,b,c}_{\bita\bitb\bitc}&=&
\cnot{\bitb}{\bita}
\cnot{\bitc}{\bitb}
\ket{a,b,c}\\
&=&
\cnot{\bitb}{\bita}\ket{a,b\oplus c,c}\\
&=&
\ket{a\oplus b \oplus c,b\oplus c,c}
\;.
\eeqa

\beqa
RHS\ket{a,b,c}_{\bita\bitb\bitc}&=&
\cnot{\bitc}{\bita}
\cnot{\bitc}{\bitb}
\cnot{\bitb}{\bita}
\ket{a,b,c} \\
&=&
\cnot{\bitc}{\bita}
\cnot{\bitc}{\bitb}
\ket{a\oplus b,b,c}\\
&=&
\cnot{\bitc}{\bita}
\ket{a\oplus b,b \oplus c,c}\\
&=&
\ket{a\oplus b \oplus c,b \oplus c,c}
\;.
\eeqa

Finally, note that
$CNOT(\bitc\rarrow\bita)$ and
$CNOT(\bitc\rarrow\bitb)
CNOT(\bitb\rarrow\bita)$ commute.
\qed

A mnemonic for remembering
Eq.(\ref{eq-com-cnots-line}):
On the left hand side of
Eq.(\ref{eq-com-cnots-line}), we have a
``chain"
CNOT($\bita\larrow\bitb$)
CNOT($\bitb\larrow\bitc$) of CNOTs.
When CNOT($\bita\larrow\bitb$) is moved
to the right (or to the left), over
CNOT($\bitb\larrow\bitc$), it leaves
behind as a ``wake"
the CNOT within the dotted box.
The wake CNOT($\bita\larrow\bitc$)
points from the beginning to the
end of the  original chain
CNOT($\bita\larrow\bitb$)
CNOT($\bitb\larrow\bitc$).

Throughout QC Paulinesia,
we will refer to equations,
like Eq.(\ref{eq-com-cnots-line}),
wherein two operators are permuted
and a wake is produced,
as ``wake identities".
Eq.(\ref{eq-com-cnots-line})
is the first of many
wake identities we will present.


\claim (Permuting 2 CNOTs in a chain,
when first and
last qubit of chain are the same)

\grayeq{
\beq
\begin{array}{c}
\Qcircuit @C=1em @R=1em @!R{
&\timesgate\qwx[1]
&\dotgate
&\qw
\\
&\dotgate
&\timesgate\qwx[-1]
&\qw
}
\end{array}
=
\begin{array}{c}
\Qcircuit @C=1em @R=1em @!R{
&\dotgate
&\timesgate\qwx[1]
&\dotgate
&\timesgate\qwx[1]
&\qw
\\
&\timesgate\qwx[-1]
&\dotgate
&\timesgate\qwx[-1]
&\dotgate
&\qw
\gategroup{2}{3}{1}{2}{.7em}{.}
}
\end{array}
\;.
\label{eq-com-cnots-loop}
\eeq}
\proof
Eq.(\ref{eq-com-cnots-loop}) is the same as

\beq
1=
\begin{array}{c}
\Qcircuit @C=1em @R=1em @!R{
&\timesgate\qwx[1]
&\dotgate
&\timesgate\qwx[1]
&\dotgate
&\timesgate\qwx[1]
&\dotgate
&\qw
\\
&\dotgate
&\timesgate\qwx[-1]
&\dotgate
&\timesgate\qwx[-1]
&\dotgate
&\timesgate\qwx[-1]
&\qw
}
\end{array}
\;,
\eeq
which is just the fact
that $E^2=1$, where $E$ is
the exchange operator.
\qed

A mnemonic for remembering
Eq.(\ref{eq-com-cnots-loop}):
On the left hand side of
Eq.(\ref{eq-com-cnots-loop}), we have a
``loop chain"
CNOT($\bita\larrow\bitb$)
CNOT($\bitb\larrow\bita$) of CNOTs.
When CNOT($\bita\larrow\bitb$) is moved over
CNOT($\bitb\larrow\bita$), it leaves
behind as a ``wake"
the two CNOTs within the dotted box.
The wake and the non-wake parts are
identical.

\claim

\grayeq{
\beq
\begin{array}{c}
\Qcircuit @C=1em @R=1em @!R{
&\dotgate\qwx[1]
&\qw
\\
&\timesgate
&\gate{\sigz}
}
\end{array}
=
\begin{array}{c}
\Qcircuit @C=1em @R=1em @!R{
&\gate{\sigz}
&\qw
&\dotgate\qwx[1]
&\qw
\\
&\qw
&\gate{\sigz}
&\timesgate
&\qw
\gategroup{1}{2}{1}{2}{.7em}{.}
}
\end{array}
\;.
\label{eq-perm-cnot-sigz}
\eeq}
(Dotted box encloses wake.)
\proof
Let LHS and RHS stand for the
left and right hand sides of
Eq.(\ref{eq-perm-cnot-sigz}).
For $a,b\in Bool$,

\beqa
LHS\ket{a,b}_{\bita\bitb}&=&
\cnot{\bita}{\bitb}\sigz(\bitb)\ket{a,b}
\\
&=& (-1)^b \ket{a, b\oplus a}
\;.
\eeqa

\beqa
RHS\ket{a,b}_{\bita\bitb}&=&
\sigz(\bita)\sigz(\bitb)
\cnot{\bita}{\bitb}\ket{a,b}
\\
&=&
\sigz(\bita)\sigz(\bitb)
\ket{a, b\oplus a}
\\
&=& (-1)^b \ket{a, b\oplus a}
\;.
\eeqa
\altproof

\beqa
\cnot{\bita}{\bitb}
\sigz(\bitb)
\cnot{\bita}{\bitb}
&=&
[\sigx(\bitb) n(\bita) + \nbar(\bita)]
\sigz(\bitb)
[\sigx(\bitb) n(\bita) + \nbar(\bita)]\\
&=&
\sigz(\bitb)
[-\sigx(\bitb) n(\bita) + \nbar(\bita)]
[\sigx(\bitb) n(\bita) + \nbar(\bita)]\\
&=&
\sigz(\bitb)[-n(\bita) + \nbar(\bita)]\\
&=&
\sigz(\bitb)\sigz(\bita)
\;.
\eeqa
\qed

\claim

\grayeq{
\beq
\begin{array}{c}
\Qcircuit @C=1em @R=1em @!R{
&\dotgate\qwx[1]
&\qw
&\dotgate\qwx[1]
&\qw
\\
&\timesgate
&\dotgate\qwx[1]
&\timesgate
&\qw
\\
&\qw
&\timesgate
&\qw
&\qw
}
\end{array}
=
\begin{array}{c}
\Qcircuit @C=1em @R=1em @!R{
&\dotgate\qwx[2]
&\qw
&\qw
\\
&\qw
&\dotgate\qwx[1]
&\qw
\\
&\timesgate
&\timesgate
&\qw
}
\end{array}
\;.
\label{eq-two-brothers}
\eeq}
\proof
Apply Eq.(\ref{eq-com-cnots-line}) once to
left hand side of Eq.(\ref{eq-two-brothers}).
\qed

Note that in Eq.(\ref{eq-two-brothers}),
the left hand side contains only nearest
neighbor CNOTs,
whereas
the right hand side
contains only commuting CNOTs.


\claim

\grayeq{
\beq
\begin{array}{c}
\Qcircuit @C=1em @R=1em @!R{
&\dotgate\qwx[1]
&\qw
&\qw
&\qw
&\dotgate\qwx[1]
&\qw
\\
&\timesgate
&\dotgate\qwx[1]
&\qw
&\dotgate\qwx[1]
&\timesgate
&\qw
\\
&\qw
&\timesgate
&\dotgate\qwx[1]
&\timesgate
&\qw
&\qw
\\
&\qw
&\qw
&\timesgate
&\qw
&\qw
&\qw
}
\end{array}
=
\begin{array}{c}
\Qcircuit @C=1em @R=1em @!R{
&\dotgate\qwx[3]
&\qw
&\qw
&\qw
\\
&\qw
&\dotgate\qwx[2]
&\qw
&\qw
\\
&\qw
&\qw
&\dotgate\qwx[1]
&\qw
\\
&\timesgate
&\timesgate
&\timesgate
&\qw
}
\end{array}
\;.
\label{eq-three-brothers}
\eeq}
\proof
Apply Eq.(\ref{eq-com-cnots-line}) twice to
left hand side of Eq.(\ref{eq-three-brothers}).
\qed

\claim

\grayeq{
\beq
\begin{array}{c}
\Qcircuit @C=1em @R=1em @!R{
&\dotgate\qwx[1]
&\qw
&\dotgate\qwx[1]
&\qw
&\qw
\\
&\timesgate
&\dotgate\qwx[1]
&\timesgate
&\dotgate\qwx[1]
&\qw
\\
&\qw
&\timesgate
&\qw
&\timesgate
&\qw
}
\end{array}
=
\begin{array}{c}
\Qcircuit @C=1em @R=1em @!R{
&\dotgate\qwx[2]
&\qw
\\
&\qw
&\qw
\\
&\timesgate
&\qw
}
\end{array}
\;.
\label{eq-cnot-next-nearest}
\eeq}
\proof
This follows immediately
from Eq.(\ref{eq-two-brothers}).
\qed

\claim

\grayeq{
\beq
\begin{array}{c}
\Qcircuit @C=1em @R=1em @!R{
&\dotgate\qwx[1]
&\qw
&\qw
&\qw
&\dotgate\qwx[1]
&\qw
&\qw
&\qw
&\qw
\\
&\timesgate
&\dotgate\qwx[1]
&\qw
&\dotgate\qwx[1]
&\timesgate
&\dotgate\qwx[1]
&\qw
&\dotgate\qwx[1]
&\qw
\\
&\qw
&\timesgate
&\dotgate\qwx[1]
&\timesgate
&\qw
&\timesgate
&\dotgate\qwx[1]
&\timesgate
&\qw
\\
&\qw
&\qw
&\timesgate
&\qw
&\qw
&\qw
&\timesgate
&\qw
&\qw
}
\end{array}
=
\begin{array}{c}
\Qcircuit @C=1em @R=1em @!R{
&\dotgate\qwx[3]
&\qw
\\
&\qw
&\qw
\\
&\qw
&\qw
\\
&\timesgate
&\qw
}
\end{array}
\;.
\label{eq-cnot-next-next-nearest}
\eeq}
\proof
The product of left hand sides
of Eqs.(\ref{eq-two-brothers})
and (\ref{eq-three-brothers}),
equals the product of their
right hand sides.
\qed

Eqs.(\ref{eq-cnot-next-nearest}) and
(\ref{eq-cnot-next-next-nearest})
suggest a way of converting a
non-nearest neighbor CNOT
into a sequence
of nearest neighbor ones.
