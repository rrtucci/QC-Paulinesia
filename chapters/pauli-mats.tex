\chapter{Pauli Matrices}


 The Pauli matrices are defined by:

\beq
\sigx =
\left(
\begin{array}{cc}
0&1\\
1&0
\end{array}
\right)
\;,\;\;
\sigy =
\left(
\begin{array}{cc}
0&-i\\
i&0
\end{array}
\right)\;,\;\;
\sigz =
\left(
\begin{array}{cc}
1&0\\
0&-1
\end{array}
\right)
\;.
\eeq
Sometimes one refers to
$\sigx,\sigy, \sigz$ as $\sigma_1, \sigma_2, \sigma_3$,
respectively. One can then use $\sigma_0$ to
denote the $2 \times 2$ identity matrix.
It is often convenient to use
the vector of Pauli matrices
$\vec{\sigma} = (\sigx, \sigy, \sigz)$.

All 3 Pauli matrices are their own inverses:
\beq
\sigx^2=\sigy^2=\sigz^2=1
\;.
\label{eq-sig-sq}
\eeq
Distinct Pauli matrices anticommute. For example,

\beq
\sigx\sigy = -\sigy\sigx
\;.
\label{eq-sig-anticom}
\eeq
It is easy to check that

\beq
\sigx\sigy = i \sigz
\;,\;\;
\sigy\sigz = i \sigx
\;,\;\;
\sigz\sigx = i \sigy
\;.
\label{eq-sigx-sigy}
\eeq
Note that Eqs.(\ref{eq-sig-sq}),
(\ref{eq-sig-anticom}) and (\ref{eq-sigx-sigy})
specify a $3 \times 3$ multiplication
table for the 3 Pauli matrices with each other.

For $w\in \{X, Y, Z\}$,
if $\ket{+_w}$ and $\ket{-_w}$ represent
the eigenvectors of
$\sigma_w$ with
eigenvalues $+1$ and $-1$, respectively,
then

\beq
\ket{+_X} =
\frac{1}{\sqrt{2}}
\left(
\begin{array}{c}
1\\1
\end{array}
\right)
\;,\;\;
\ket{-_X} =
\frac{1}{\sqrt{2}}
\left(
\begin{array}{c}
1\\-1
\end{array}
\right)
\;,
\eeq

\beq
\ket{+_Y} =
\frac{1}{\sqrt{2}}
\left(
\begin{array}{c}
1\\i
\end{array}
\right)
\;,\;\;
\ket{-_Y} =
\frac{1}{\sqrt{2}}
\left(
\begin{array}{c}
1\\-i
\end{array}
\right)
\;,
\eeq

\beq
\ket{+_Z} =
\left(
\begin{array}{c}
1\\0
\end{array}
\right)
\;,\;\;
\ket{-_Z} =
\left(
\begin{array}{c}
0\\1
\end{array}
\right)
\;.
\eeq


We define

\beq
\ket{0} =\ket{+_Z}
\;,
\eeq
and

\beq
\ket{1} =\ket{-_Z}
\;.
\eeq
We will use
$n$ to denote the ``number operator". Thus,


\beq
n =  \left(\begin{array}{cc}
0&0\\
0&1
\end{array}\right)
=\ket{-_Z}\bra{-_Z}=  \frac{1-\sigma_Z}{2}
\;,
\eeq
and

\beq
\nbar = 1-n = \left(\begin{array}{cc}
1&0\\
0&0
\end{array}\right)
=\ket{+_Z}\bra{+_Z}=  \frac{1+\sigma_Z}{2}
\;.
\eeq
Since $n$ and $\sigma_Z$ are diagonal, it is
easy to see that

\beq
\sigma_Z =(-1)^n = 1 - 2n
\;.
\eeq

Most of the definitions and results stated so far
for $\sigz$
have counterparts for
$\sigz$ and $\sigy$.
The counterpart results can be
easily proven by applying a rotation that
interchanges the coordinate axes.
Let $w\in \{X, Y, Z\}$.
If $\ket{+_w}$ and $\ket{-_w}$
represent the eigenvectors of $\sigma_w$ with
eigenvalues $+1$ and $-1$, respectively,
then we define

\beq
\ket{0_w} =\ket{+_w}
\;,
\eeq
and

\beq
\ket{1_w} =\ket{-_w}
\;.
\eeq
Let


\beq
n_w=
\ket{-_w}\bra{-_w}=  \frac{1-\sigma_w}{2}
\;,
\eeq

\beq
\nbar_w = 1-n_w
=\ket{+_w}\bra{+_w}=  \frac{1+\sigma_w}{2}
\;.
\eeq
As when $w=Z$, one has

\beq
 \sigma_w = (-1)^{n_w} = 1 - 2n_w
\;.
\eeq


Note that whenever we use
$\ket{0}$, $\ket{1}$ or $n$ ,
without an $X, Y$ or $Z$ subscript,
the subscript $Z$ should be inferred.

The one bit Hadamard matrix is defined by:

\beq
H = \frac{1}{\sqrt{2}}
\left(
\begin{array}{cc}
1&1\\
1&-1
\end{array}
\right)=
\frac{1}{\sqrt{2}}
(\sigx + \sigz)
\;.
\eeq
It is easy to check that

\beq
H^2 =  1
\;,
\eeq

\beq
H\sigx H = \sigz\;,\;\;
H\sigz H = \sigx
\;,
\eeq

\beq
\ket{0_X} =
\frac{\ket{0} + \ket{1}}{\sqrt{2}} =
H \ket{0}
\;,
\eeq

\beq
\ket{1_X} =
\frac{\ket{0} - \ket{1}}{\sqrt{2}} =
H \ket{1}
\;.
\eeq

The matrix $i^n$ is defined by

\beq
i^n =
\left(
\begin{array}{cc}
1&0\\
0&i
\end{array}
\right)
\;.
\eeq
It is easy to check that

\beq
(i^n)^2 = \sigz
\;,
\eeq

\beq
i^n\sigx i^{-n} = \sigy \;,\;\;
i^{-n}\sigx i^{n} = -\sigy
\;.
\eeq

Note that for $a,b\in Bool$,

\beq
\sigx^b\ket{a} = \ket{a\oplus b}
\;,
\eeq

\beq
\sigz^b\ket{a} = (-1)^{ab}\ket{a}
\;,
\eeq

\beq
\bra{a}H\ket{b} = \frac{(-1)^{ab}}{\sqrt{2}}
\;.
\eeq


A general qubit rotation is defined by
$e^{i\vec{\theta}\cdot \vec{\sigma}}$,
where $\vec{\theta}$ is a 3 dimensional
real vector.
For any real
number $\theta$,

\beq
e^{i\theta\sigz} =
\cos\theta + i \sigz \sin\theta
\;.
\label{eq-z-rot}
\eeq
Eq.(\ref{eq-z-rot}) can
be proven by expressing both sides
of it as a
power series.
Applying a rotation to
Eq.(\ref{eq-z-rot}), it becomes

\beq
e^{i\vec{\theta}\cdot \vec{\sigma}}=
\cos\theta +
i\vec{\sigma}\cdot\hat{\theta}
\sin\theta
\;,
\eeq
where $\vec{\theta}$ is a
3 dimensional real vector,
$\theta$ is its magnitude, and
$\hat{\theta}=\vec{\theta}/\theta$.
