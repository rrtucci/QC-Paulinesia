
\chapter{CNOT Generalizations}

In this section, $\vec{\bita}$,
$\vec{\bitb}$ and $\vec{\bitc}$
will denote disjoint
sets of distinct qubits.
That is,
any two different components of the same
vector, or two components of
different vectors
represent different qubits.

Suppose $U$ is a unitary matrix.
Furthermore,
for $j=1,2$, suppose
$\pi_j$
is a projection operator
(i.e., $\pi_j^2 = \pi_j$,
the eigenvalues of $\pi_j$
are all 0 or 1).
Some examples of projection operators
$\pi_j$ that
are of interest to us: 1, $n(\bita)$,
$n(\bita)n(\bitb)$,
$n(\bita)\nbar(\bitb)$,
$n(\bita)n(\bitb)n(\bitc)$, etc.
It is convenient to generalize CNOT
diagrammatic notation as follows.
Let

\grayeq{
\beq
\begin{array}{c}
\Qcircuit @C=1em @R=1em @!R{
&\ovalgate{\pi_1}\qwx[1]
&\rstick{\vec{\bita}}\qw
\\
&\ovalgate{\pi_2}
&\rstick{\vec{\bitb}}\qw
}
\end{array}
\;\;\;=
(-1)^{\pi_1(\vec{\bita})\pi_2(\vec{\bitb})}
\;,
\label{eq-oval-oval}
\eeq}
and

\grayeq{
\beq
\begin{array}{c}
\Qcircuit @C=1em @R=1em @!R{
&\ovalgate{\pi_1}\qwx[1]
&\rstick{\vec{\bita}}\qw
\\
&\gate{U}
&\rstick{\vec{\bitb}}\qw
}
\end{array}
\;\;\;=
{U(\vec{\bitb})}^{\pi_1(\vec{\bita})}
\;.
\label{eq-oval-sq}
\eeq}
We will refer to an operator of the form
Eq.(\ref{eq-oval-sq})
as a {\bf projector controlled unitary operator},
or simply as a {\bf controlled U},
in analogy to a controlled NOT,
for which $U=\sigx=$ the NOT operator.
The set of operators of the form
Eq.(\ref{eq-oval-oval})
is a subset of the set of operators of the form
Eq.(\ref{eq-oval-sq}). Indeed,
given any projection operator
$\pi_2(\vec{\bitb})$, one can always define
the unitary operator
$U(\vec{\bitb})
 = (-1)^{\pi_2(\vec{\bitb})}
= 1 - 2\pi_2(\vec{\bitb})$. Hence,


\beq
\begin{array}{c}
\Qcircuit @C=1em @R=1em @!R{
&\ovalgate{\pi_1}\qwx[1]
&\qw
\\
&\gate{(-1)^{\pi_2}}
&\qw
}
\end{array}
=
\begin{array}{c}
\Qcircuit @C=1em @R=1em @!R{
&\ovalgate{\pi_1}\qwx[1]
&\qw
\\
&\ovalgate{\pi_2}
&\qw
}
\end{array}
\;.
\eeq
Special cases of
Eqs.(\ref{eq-oval-oval}) and
(\ref{eq-oval-sq}) are:

\beq
(-1)^{n(\bita)n(\bitb)}
=
\begin{array}{c}
\Qcircuit @C=1em @R=1em @!R{
&\dotgate\qwx[1]
&\qw
\\
&\dotgate
&\qw
}
\end{array}
=
\begin{array}{c}
\Qcircuit @C=1em @R=1em @!R{
&\ovalgate{n}\qwx[1]
&\qw
\\
&\ovalgate{n}
&\qw
}
\end{array}
=
\begin{array}{c}
\Qcircuit @C=1em @R=1em @!R{
&\ovalgate{n}\qwx[1]
&\qw
\\
&\gate{\sigz}
&\qw
}
\end{array}
\;,
\eeq

\beq
\cnot{\bita}{\bitb}
=
\begin{array}{c}
\Qcircuit @C=1em @R=1em @!R{
&\dotgate\qwx[1]
&\qw
\\
&\timesgate
&\qw
}
\end{array}
=
\begin{array}{c}
\Qcircuit @C=1em @R=1em @!R{
&\ovalgate{n}\qwx[1]
&\qw
\\
&\ovalgate{n_X}
&\qw
}
\end{array}
=
\begin{array}{c}
\Qcircuit @C=1em @R=1em @!R{
&\ovalgate{n}\qwx[1]
&\qw
\\
&\gate{\sigx}
&\qw
}
\end{array}
\;,
\eeq
and, for any $2\times 2$ unitary matrix $U$:

\beq
U(\bitb)^{n(\bita)}
=
\begin{array}{c}
\Qcircuit @C=1em @R=1em @!R{
&\dotgate\qwx[1]
&\qw
\\
&\gate{U}
&\qw
}
\end{array}
=
\begin{array}{c}
\Qcircuit @C=1em @R=1em @!R{
&\ovalgate{n}\qwx[1]
&\qw
\\
&\gate{U}
&\qw
}
\end{array}
\;,
\label{eq-n-one-cu}
\eeq
\beq
U(\bitc)^{n(\bita)n(\bitb)}
=
\begin{array}{c}
\Qcircuit @C=1em @R=.5em @!R{
&\dotgate\qwx[2]
&\qw
\\
&\dotgate
&\qw
\\
&\gate{U}
&\qw
}
\end{array}
=
\begin{array}{c}
\Qcircuit @C=1em @R=.5em @!R{
&\ovalgate{n}\qwx[1]
&\qw
\\
&\ovalgate{n}\qwx[1]
&\qw
\\
&\gate{U}
&\qw
}
\end{array}
\;.
\label{eq-n-two-cu}
\eeq
We will refer to the
operator of Eq.(\ref{eq-n-one-cu})
as an $n^1$ {\bf controlled U},
and to the operator of
Eq.(\ref{eq-n-two-cu})
as an $n^2$ {\bf controlled U}.

Suppose $U$ is any $2\times 2$ unitary matrix.
It can always be diagonalized as follows:

\beq
U = V diag(
e^{i\theta_1},
e^{i\theta_2})
V^\dagger
\;,
\eeq
where $\theta_1, \theta_2$
are reals numbers and
$V$ is a unitary matrix. If we set

\beq
\Delta = \frac{\theta_1-\theta_2}{2}
\;,
\eeq
and

\beq
\overline{\theta} = \frac{\theta_1+\theta_2}{2}
\;,
\eeq
then

\beq
U= e^{i\overline{\theta}}
V e^{i\Delta \sigz}
V^\dagger
\;.
\label{eq-u-eigen-decomp}
\eeq


\claim

For any $2\times 2$ unitary matrix
$U(\bitb)$ given by Eq.(\ref{eq-u-eigen-decomp}),
and projection operator $\pi_1(\vecbita)$,

\grayeq{
\beq
\begin{array}{c}
\Qcircuit @C=1em @R=1em @!R{
&\ovalgate{\pi_1}\qwx[1]
&\qw
\\
&\gate{U}
&\qw
}
\end{array}
=
\begin{array}{c}
\Qcircuit @C=1em @R=1em @!R{
&\gate{e^{i\overline{\theta}\pi_1}}
&\qw
&\ovalgate{\pi_1}\qwx[1]
&\qw
&\ovalgate{\pi_1}\qwx[1]
&\qw
&\rstick{\vec{\bita}}\qw
\\
&\gate{V}
&\gate{e^{i\frac{\Delta}{2}\sigz}}
&\timesgate
&\gate{e^{-i\frac{\Delta}{2}\sigz}}
&\timesgate
&\gate{V^\dagger}
&\rstick{\bitb}\qw
}
\end{array}
\;\;\;\;.
\label{eq-pi-contr-u-decompo}
\eeq}
\proof
Check that both sides agree when $\pi_1$
equals 0 and 1.
\altproof
\beq
U(\bitb)^{n(\bita)}=
e^{i\overline{\theta}n(\bita)}
V(\bitb) e^{i\Delta \sigz(\bitb)n(\bita)}
V(\bitb)^\dagger
\;.
\eeq

\beqa
e^{i\Delta \sigz(\bitb)n(\bita)}
&=&
e^{i\Delta \sigz(\bitb)
\frac{1}{2}[1-\sigz(\bita)]}
\\
&=&
e^{i \frac{\Delta}{2}\sigz(\bitb)}
e^{-i \frac{\Delta}{2}
\sigz(\bitb)
\sigz(\bita)}
\\
&=&
e^{i \frac{\Delta}{2}\sigz(\bitb)}
\sigx(\bitb)^{n(\bita)}
e^{-i \frac{\Delta}{2}
\sigz(\bitb)}
\sigx(\bitb)^{n(\bita)}
\;.
\eeqa
This proof still holds if we  replace $n(\bita)$
by $\pi_1(\vec{\bita})$ and $\sigz(\bita)$
by $(-1)^{\pi_1(\vec{\bita})}$.
\qed

Examples of Eq.(\ref{eq-pi-contr-u-decompo})
are:

\grayeq{
\beq
\begin{array}{c}
\Qcircuit @C=1em @R=.5em @!R{
&\dotgate\qwx[1]
&\qw
\\
&\gate{U}
&\qw
}
\end{array}
=
\begin{array}{c}
\Qcircuit @C=1em @R=.5em @!R{
&\gate{e^{i\bar{\theta}n}}
&\qw
&\dotgate\qwx[1]
&\qw
&\dotgate\qwx[1]
&\qw
&\qw
\\
&\gate{V}
&\gate{e^{i\frac{\Delta}{2}\sigz}}
&\timesgate
&\gate{e^{-i\frac{\Delta}{2}\sigz}}
&\timesgate
&\gate{V^\dagger}
&\qw
}
\end{array}
\;,
\label{eq-n-one-contr-u}
\eeq}
and


\grayeq{
\beq
\begin{array}{c}
\Qcircuit @C=1em @R=.5em @!R{
&\dotgate\qwx[1]
&\qw
\\
&\dotgate\qwx[1]
&\qw
\\
&\gate{U}
&\qw
}
\end{array}
=
\begin{array}{c}
\Qcircuit @C=1em @R=.5em @!R{
&\dotgate\qwx[1]
&\qw
&\dotgate\qwx[1]
&\qw
&\dotgate\qwx[1]
&\qw
&\qw
\\
&\gate{e^{i\bar{\theta}n}}
&\qw
&\dotgate\qwx[1]
&\qw
&\dotgate\qwx[1]
&\qw
&\qw
\\
&\gate{V}
&\gate{e^{i\frac{\Delta}{2}\sigz}}
&\timesgate
&\gate{e^{-i\frac{\Delta}{2}\sigz}}
&\timesgate
&\gate{V^\dagger}
&\qw
}
\end{array}
\;.
\label{eq-n-two-contr-u}
\eeq}
Eqs.(\ref{eq-n-one-contr-u}) and
(\ref{eq-n-two-contr-u}) suggest a way
of converting any $n^r$ controlled
$U$, for an integer $r\geq 1$,
into a sequence of gates containing no
controlled $U$'s but containing
$n^s$ controlled NOTs,
where $s\leq r$.

\claim(Permuting two projector
controlled $U$'s)

Suppose $\pi_1(\vecbita), \pi_2(\vecbita)$
are  commuting
($[\pi_1, \pi_2]=0$)
projection operators
and $U_1(\vecbitb), U_2(\vecbitb)$
are unitary operators. Then

\grayeq{
\beq
\begin{array}{c}
\Qcircuit @C=1em @R=.5em @!R{
&\ovalgate{\pi_1}\qwx[1]
&\ovalgate{\pi_2}\qwx[1]
&\qw
\\
&\gate{U_1}
&\gate{U_2}
&\qw
}
\end{array}
=
\begin{array}{c}
\Qcircuit @C=1em @R=.5em @!R{
&\ovalgate{\pi_1\pi_2}\qwx[1]
&\ovalgate{\pi_2}\qwx[1]
&\ovalgate{\pi_1}\qwx[1]
&\rstick{\vecbita}\qw
\\
&\gate{U_1 U_2 U_1^\dagger U_2^\dagger}
&\gate{U_2}
&\gate{U_1}
&\rstick{\vecbitb}\qw
\gategroup{1}{2}{2}{2}{.7em}{.}
}
\end{array}
\;\;\;\;.
\label{eq-perm-two-pi-contr-u}
\eeq}
(Dotted box encloses wake.)
Algebraically,

\beq
U_1(\vecbitb)^{\pi_1(\vecbita)}
U_2(\vecbitb)^{\pi_2(\vecbita)}
=
(U_1 U_2 U^\dagger_1 U^\dagger_2)^{\pi_1\pi_2}
U_2^{\pi_2}U_1^{\pi_1}
\;.
\eeq
\proof
Check that both sides of
Eq.(\ref{eq-perm-two-pi-contr-u})
agree when $(\pi_1,\pi_2)$
equals each element of $Bool^2$.
\qed


\claim

For any projection operator
$\pi_1(\vec{\bita})$ and
unitary matrix $U(\vec{\bitc})$,


\grayeq{
\beq
\begin{array}{c}
\Qcircuit @C=1em @R=.5em @!R{
&\ovalgate{\pi_1}\qwx[1]
&\qw
&\qw
\\
&\timesgate
&\dotgate\qwx[1]
&\qw
\\
&\qw
&\gate{U}
&\qw
}
\end{array}
=
\begin{array}{c}
\Qcircuit @C=1em @R=.5em @!R{
&\ovalgate{\pi_1}\qwx[2]
&\ovalgate{\pi_1}\qwx[2]
&\qw
&\ovalgate{\pi_1}\qwx[1]
&\rstick{\vec{\bita}}\qw
\\
&\dotgate
&\qw
&\dotgate\qwx[1]
&\timesgate
&\rstick{\bitb}\qw
\\
&\gate{U^{-2}}
&\gate{U}
&\gate{U}
&\qw
&\rstick{\vec{\bitc}}\qw
\gategroup{1}{2}{3}{3}{.7em}{.}
}
\end{array}
\;\;\;\;.
\label{eq-perm-gen-times-dot}
\eeq}
(Dotted box encloses wake.)
\proof
Consider Eq.(\ref{eq-perm-two-pi-contr-u})
with the following replacements:
$U_1\rarrow \sigx(\bitb)$,
$U_2\rarrow U(\vecbitc)^{n(\bitb)}$,
$\pi_2\rarrow 1$.
Thus,

\beq
U_1 U_2 U_1^\dagger U_2^\dagger
\rarrow
\sigx(\bitb)
U(\vecbitc)^{n(\bitb)}
\sigx(\bitb)
U(\vecbitc)^{-n(\bitb)}
=U(\vecbitc)^{\nbar(\bitb)-n(\bitb)}
=U(\vecbitc)^{1-2n(\bitb)}
\;.
\eeq
\qed

\claim

For any projection operator
$\pi_1(\vec{\bita})$ and
unitary matrix $U(\vec{\bitc})$,

\grayeq{
\beq
\begin{array}{c}
\Qcircuit @C=1em @R=.5em @!R{
&\ovalgate{\pi_1}\qwx[2]
&\qw
\\
&\dotgate
&\qw
\\
&\gate{U}
&\qw
}
\end{array}
=
\begin{array}{c}
\Qcircuit @C=1em @R=.5em @!R{
&\ovalgate{\pi_1}\qwx[2]
&\ovalgate{\pi_1}\qwx[1]
&\qw
&\ovalgate{\pi_1}\qwx[1]
&\qw
&\rstick{\vec{\bita}}\qw
\\
&\qw
&\timesgate
&\dotgate\qwx[1]
&\timesgate
&\dotgate\qwx[1]
&\rstick{\bitb}\qw
\\
&\gate{U^{\frac{1}{2}}}
&\qw
&\gate{U^{\frac{-1}{2}}}
&\qw
&\gate{U^{\frac{1}{2}}}
&\rstick{\vec{\bitc}}\qw
}
\end{array}
\;\;\;.
\label{eq-pi-n-contr-u}
\eeq}
\proof
Apply Eq.(\ref{eq-perm-gen-times-dot})
to the right hand side of
Eq.(\ref{eq-pi-n-contr-u}) to permute
$\sigx(\bitb)^{\pi_1(\vecbita)}$
and
$U(\vecbitc)^{\frac{-1}{2}n(\bitb)}$.
\qed

Examples of Eq.(\ref{eq-pi-n-contr-u}) are

\grayeq{
\beq
\begin{array}{c}
\Qcircuit @C=1em @R=.5em @!R{
&\dotgate\qwx[2]
&\qw
\\
&\dotgate
&\qw
\\
&\gate{U}
&\qw
}
\end{array}
=
\begin{array}{c}
\Qcircuit @C=1em @R=.5em @!R{
&\dotgate\qwx[2]
&\dotgate\qwx[1]
&\qw
&\dotgate\qwx[1]
&\qw
&\qw
\\
&\qw
&\timesgate
&\dotgate\qwx[1]
&\timesgate
&\dotgate\qwx[1]
&\qw
\\
&\gate{U^{\frac{1}{2}}}
&\qw
&\gate{U^{\frac{-1}{2}}}
&\qw
&\gate{U^{\frac{1}{2}}}
&\qw
}
\end{array}
\;,
\label{eq-n-two-sq-root-u}
\eeq}
and

\grayeq{
\beq
\begin{array}{c}
\Qcircuit @C=1em @R=.5em @!R{
&\dotgate\qwx[1]
&\qw
\\
&\dotgate\qwx[2]
&\qw
\\
&\dotgate
&\qw
\\
&\gate{U}
&\qw
}
\end{array}
=
\begin{array}{c}
\Qcircuit @C=1em @R=.5em @!R{
&\dotgate\qwx[1]
&\dotgate\qwx[1]
&\qw
&\dotgate\qwx[1]
&\qw
&\qw
\\
&\dotgate\qwx[2]
&\dotgate\qwx[1]
&\qw
&\dotgate\qwx[1]
&\qw
&\qw
\\
&\qw
&\timesgate
&\dotgate\qwx[1]
&\timesgate
&\dotgate\qwx[1]
&\qw
\\
&\gate{U^{\frac{1}{2}}}
&\qw
&\gate{U^{\frac{-1}{2}}}
&\qw
&\gate{U^{\frac{1}{2}}}
&\qw
}
\end{array}
\;.
\label{eq-n-three-sq-root-u}
\eeq}

Eqs.(\ref{eq-n-two-sq-root-u})
and (\ref{eq-n-three-sq-root-u})
suggest a way of converting
an $n^r$ controlled $U$,
for an integer $r\geq 2$,
into a sequence of gates
that contains no controlled $U$'s
except $n^1$ controlled $U$'s.

\claim

Suppose $\pi_1(\vecbita)$
and $\pi_2(\vecbita)$ are
commuting projection operators. Then

\grayeq{
\beq
\begin{array}{c}
\Qcircuit @C=1em @R=1em @!R{
&\ovalgate{\pi_1}\qwx[1]
&\ovalgate{\pi_2}\qwx[1]
&\qw
\\
&\timesgate
&\dotgate
&\qw
}
\end{array}
=
\begin{array}{c}
\Qcircuit @C=1em @R=1em @!R{
&\gate{(-1)^{\pi_1\pi_2}}
&\ovalgate{\pi_2}\qwx[1]
&\ovalgate{\pi_1}\qwx[1]
&\rstick{\vec{\bita}}\qw
\\
&\qw
&\dotgate
&\timesgate
&\rstick{\bitb}\qw
\gategroup{1}{2}{1}{2}{.7em}{.}
}
\end{array}
\;\;\;.
\label{eq-perm-gen-chain}
\eeq}
(Dotted box encloses wake.)
\proof
Consider Eq.(\ref{eq-perm-two-pi-contr-u})
with the following replacements:
$U_1\rarrow \sigx(\bitb)$,
$U_2\rarrow \sigz(\bitb)$.
Thus,

\beq
U_1 U_2 U_1^\dagger U_2^\dagger
\rarrow
\sigx\sigz\sigx\sigz = -1
\;.
\eeq
\qed


Eq.(\ref{eq-perm-gen-chain})
can be used to transform
sequences of $n^r$ controlled
NOTs. For example,
the following identity
can be easily proven by applying
Eq.(\ref{eq-perm-gen-chain}):

\beq
\begin{array}{c}
\Qcircuit @C=1em @R=1em @!R{
&\qw
&\dotgate\qwx[4]
&\qw
&\dotgate\qwx[4]
&\qw
\\
&\dotgate\qwx[2]
&\qw
&\dotgate\qwx[2]
&\qw
&\qw
\\
&\dotgate
&\qw
&\dotgate
&\qw
&\qw
\\
&\timesgate
&\dotgate
&\timesgate
&\dotgate
&\qw
\\
&\qw
&\timesgate
&\qw
&\timesgate
&\qw
}
\end{array}
=
\begin{array}{c}
\Qcircuit @C=1em @R=1em @!R{
&\dotgate\qwx[4]
&\qw
\\
&\dotgate
&\qw
\\
&\dotgate
&\qw
\\
&\qw
&\qw
\\
&\timesgate
&\qw
}
\end{array}
\;.
\label{eq-n-two-to-n-three}
\eeq
Note that Eq.(\ref{eq-n-two-to-n-three})
reduces an $n^3$ controlled NOT
into a sequence of $n^2$ controlled NOTs.

\claim

For any real number $\theta$,

\grayeq{
\beq
\begin{array}{c}
\Qcircuit @C=1em @R=1em @!R{
&\ovalgate{\pi_1}\qwx[1]
&\qw
&\qw
\\
&\timesgate
&\gate{e^{i\theta\sigz}}
&\qw
}
\end{array}
=
\begin{array}{c}
\Qcircuit @C=1em @R=1em @!R{
&\gate{\pi_1}\qwx[1]
&\qw
&\gate{\pi_1}\qwx[1]
&\rstick{\vec{\bita}}\qw
\\
&\gate{e^{-2i\theta\sigz}}
&\gate{e^{i\theta\sigz}}
&\timesgate
&\rstick{\bitb}\qw
\gategroup{1}{2}{2}{2}{.7em}{.}
}
\end{array}
\;\;\;.
\eeq}
(Dotted box encloses wake.)
\proof
Consider Eq.(\ref{eq-perm-two-pi-contr-u})
with the following replacements:
$U_1\rarrow \sigx(\bitb)$,
$U_2\rarrow e^{i\theta \sigz(\bitb)}$,
$\pi_2\rarrow 1$.
Thus,

\beq
U_1 U_2 U_1^\dagger U_2^\dagger
\rarrow
\sigx(\bitb)
e^{i\theta \sigz(\bitb)}
\sigx(\bitb)
e^{-i\theta \sigz(\bitb)}
=e^{-2i\theta \sigz(\bitb)}
\;.
\eeq
\qed
